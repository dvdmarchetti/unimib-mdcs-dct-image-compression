\documentclass[12pt]{article}
\usepackage{blindtext}
\usepackage{imakeidx}
\usepackage[utf8]{inputenc}
\usepackage[T1]{fontenc}
\usepackage{enumitem}

\title{%
  MCS 2020 \\
  \large Algebra lineare numerica \\
   Compressione di immagini tramite la DCT}

\date{21/05/2020}
\author{Silva Edoardo 816560, Zhigui Bryan 816335, Marchetti Davide 815990}
\makeindex

\begin{document}
\maketitle


\section*{Abstract}
 	Si vuole avere un confronto dei tempi d’esecuzione della DCT2 della libreria scelta con la nostra implementazione. I problemi da risolvere sono delle matrici quadrate nella quale si definisce un offset di partenza che definirà la matrice iniziale e quelle avvenire fino a una dimensione massima denotata di default. 
I risultati ottenuti saranno riportati graficamente in una scala semi-algoritmica.

\section{Librerie}
	\subsection{Calcoli matematici}
	
Il pacchetto \textbf{Numpy} è una libreria open source che contiene diverse funzioni e metodi utili per il calcolo scientifico. In particolar modo, per il calcolo vettoriale e matrici multidimensionale in maniera efficiente e veloce.\newline
Il pacchetto \textbf{SciPy} è una libreria open source che utilizza la funzionalità di Numpy per fornire un pacchetto di calcolo scientifico General-purpose. In particolare, in questo progetto si utilizza la estensione Fast Fourier Transform (fftpack) per lavorare con le dct e dctn(dct2).

	\subsection{Grafici}
	
	Il pacchetto \textbf{Pandas} consente la manipolazione di dati in formato tabellare o sequenziale.\newline
	Il pacchetto \textbf{Matplotlib} consente la generazione di grafici 2D. è costituita da 3 componenti importanti:
	\begin{enumerate}
		\item La \textbf{matplotlib} API: viene utilizzata per inserire funzionalità per la creazione di grafici nei propri script python
		\item Il modulo \textbf{pyplot}: è una interfaccia in relazione con la componente precedente ed è progettato per emulare le stesse funzionalità grafiche rispetto a MATLAB.
		\item Ouput back-end: produce un output dei grafici su varie tipologie di GUI e su diversi formati di file
	\end{enumerate}
	Il pacchetto \textbf{Seaborn} serve per la creazione di grafici statistici. Legato a Matplotib e strettamente integrato con le strutture dei dati Pandas.
	
	
\section{Funzione DCT2 custom}
	
	La DCT2 è stata creata grazie all’utilizzo della DCT che viene applicata inizialmente sulle colonne e successivamente sulle righe. Inoltre, sapendo che lavora solo su vettori si è utilizzata la libreria numpy.
	
	\subsection{Funzione DCT2}
	
	Si ha in input una matrice M di dimensione NxN sulla quale viene creata una secondaria C con le stesse dimensioni ma inizializzata tutta a 0.\newline
Inizialmente si effettua la DCT sui vettori colonna della matrice originale salvando i risultati ottenuti in colonna sulla matrice C. \newline
Successivamente, viene applicata la DCT sui vettori riga della matrice C sovrascrivendo i risultati ottenuti sulle righe della stessa matrice.
Infine, in output risiede la nuova Matrice C.
	
	\subsection{Funzioni DCT e IDCT}

	La funzione DCT prende in input un vettore V di dimensione n avente come valori i coefficienti f(k).
Viene creato un vettore C di dimensione n inizializzata tutta a 0.
Ogni coefficiente letto viene applicata la formula (come spiegata a lezione) ricavando così i coefficienti c(k) che saranno salvati tutti sul vettore C dato in output.
\newline
Viene effettuato lo stesso flusso rispetto alla IDCT.

\section{Oggetto custom python(Dizionario?)}

	Si è deciso di creare un oggetto customizzato in modo tale da avere i risultati in un formatto standard per ogni problema risolto dalle due funzioni messe a 		confronto.\newline
	L’oggetto ha le seguenti caratteristiche:
	\begin{itemize}
	\item È una matrice principale che contiene un numero di record = 2*(\#problemi\_risolti). 
	\item Ogni vettore riporta le seguenti informazioni:
		\begin{itemize}
		\item Dimensione della matrice risolta (o problema);
		\item Tempo di esecuzione: che si calcola grazie alla libreria time di python (time.perf\_counter());
		\item Tipo di libreria usata: scipy oppure custom.	
		\end{itemize}
	\end{itemize}

\section{Creazione grafico}

	Viene creata una figura avente due assi cartesiani con:
	\begin{itemize}
		\item Le ordinate che rappresentano i tempi di esecuzione in scala logaritmica;
		\item Le ascisse che riportano la dimensione dei problemi risolti.
	\end{itemize}
	I dati pressi dall’oggetto custom vengono trasformati in Dataframe mediante la libreria \textbf{Pandas}.\newline
Ogni dataframe viene rappresentato come una “pallina” nel grafico e viene congiunto agli altri mediante una linea. Da notare che i risultati vengono differenziati rispetto al tipo (implemented oppure scipy) e poi viene visualizzato tutto graficamente con l’utilizzo della libreria Seasbon.
	
\section{Conclusioni}
	Dallo studio grafico possiamo notare che la DCT2 customizzata ha un tempo di esecuzione $O(N^3)$ infatti nel grafico continua a crescere la linea mentre la DCT2 della libreria Scipy ha un tempo di esecuzione O($N^2logN$) ovvero migliore alla nostra.\newline
C’è da considerare alcuni aspetti però:
	\begin{itemize}
		\item Noi utilizziamo una DCT2 che si bassa sull’esecuzione della DCT per ben due volte e questo porta a un radico rallentamento del tempo di esecuzione, basandoci sulle formule basilari.
		\item La DCT2 della libreria oltre al fatto di effettuare un’operazione del tutto compatta utilizza delle formule matematiche (FAST FOURIER) che ne permette il calcolo molto più veloce.
	\end{itemize}

\end{document}