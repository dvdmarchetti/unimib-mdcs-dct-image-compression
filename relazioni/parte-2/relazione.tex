\documentclass[12pt]{article}
\usepackage{blindtext}
\usepackage{imakeidx}
\usepackage[utf8]{inputenc}
\usepackage[T1]{fontenc}
\usepackage{enumitem}
\usepackage[obeyspaces]{url}
\usepackage{graphicx}
\graphicspath{ }

\title{%
  MCS 2020 \\
  \large Algebra lineare numerica \\
   Compressione di immagini tramite la DCT}

\date{21/05/2020}
\author{Silva Edoardo 816560, Zhigui Bryan 816335, Marchetti Davide 815990}
\makeindex

\begin{document}
\maketitle

\section{Abstract}

	Si vuole presentare un software che implementa una compressione delle immagini in modo tale che esegua i task descritti nella traccia.
	\newline
	Il software permette all’utente:
	\begin{enumerate}
		\item Scegliere dal filesystem un’immagine .bmp in toni di grigio riportando un messaggio d’errore qual ora scegliesse un formato diverso oppure a colori.
	 	\item Scegliere un valore intero F che determina il valore massimo che può scegliere del valore d che né indicherà la soglia di taglio delle frequenze.
 		\item Visualizzare sullo schermo affiancante: l’immagine originale e l’immagine ottenuta post compressione.
		\item Ridimensionare l’immagine per riempire l’area mancante rispetto a quella originale.
	\end{enumerate}

Il software è stato scritto in python per lo sviluppo della GUI moderna con PyQT5.
\newpage
\section{Implementazione}

	Il software inizialmente presenta un’interfaccia dotate con le seguenti caratteristiche:
	\begin{enumerate}
		\item Un pannello diviso a metà dedicando una sezione all’immagine originale e una sezione all’immagine compressa;
		\item Una sezione di parametri nella quale sono suddivise in due parti:
		\begin{enumerate}[label=\Alph*]
			\item \textbf{Input:} permette di andare a reperire l’immagine e scalarla alle dimensioni disponibili con l’apposito check-box
			\item \textbf{Parameters:} definisce i parametri sulla qualità di compressione dell’immagine.
		\end{enumerate}
	\end{enumerate}
	Il programma in \textbf{back-end} va a connettersi ai singoli eventi che deve avviare mano a mano che l’utente decide di effettuare un’operazione settando o facendo sparire alcune informazioni non più rilevanti.
	%immagine%

\section{Libreria PyQT5}

PyQT5 è una libreria che consente di usare il framework QT5 GUI che serve per creare GUI application nel linguaggio C++.\newline
PyQT5 è un toolkit multipiattaforma che può essere eseguito su quasi tutti i sistemi operativi.\newline
Usandolo con Python, è possibile creare applicazioni molto rapidamente senza perdere gran parte della velocità del C++\newline
I moduli utilizzati in particolare sono:
	\begin{itemize}
		\item\textbf{QtWidgets}
	 	\item\textbf{QtCore}
	 	\item\textbf{QtGUI}
	 	\item\textbf{QtWidgets}
	\end{itemize}

\section{Richiesta immagine}

	Si richiede all’utente di andare a reperire un’immagine cliccando sul bottone “open file” che presenta inizialmente la cartella dove si trova il progetto perché contiene una cartella apposita \path{..\folder\resources} dove può reperire immagini di esempio, ma può anche sceglierne una in locale del suo pc.\newline
	Una volta presa l’immagine va a verificare:
	\begin{enumerate}
		\item Se il formatto è .bmp, altrimenti presenta in messagge-box-error indicando che il formato dell’immagine è sbagliato;
		\item Se l’immagine passata non sia a colori presentando un messagge-box-error indicando che l’immagine deve essere in toni di grigio.
	\end{enumerate}
	\includegraphics{immagine}

	L’immagine viene caricata nella prima sezione del pannello permettendo di scalarla come in figura ….
	%immagine%

\section{Compressione immagine}
	Una volta che viene caricata l’immagine l’utente dovrà definire i valori di F e d in modo tale da studiare la qualità della compressione con diversi esempi che elencheremo alla fine.\newline
	Si spiega l’algoritmo nei seguenti passi:
	\begin{enumerate}
			\item L’immagine originale viene trasformata in formato pixmap;
			\item Si definisce il numero totale di bit in rapporto con il numero di canali (RGBA= 4 canali);
			\item Setta i valori di F e d immessi dall’utente;
				\begin{enumerate}[label=\Alph*]
					\item Dato F si definisce l’ampiezza dei macro-blocchi;
				\end{enumerate}
			\item Quindi l’immagine viene divisa in blocchi f di pixels di dimensione F x F, partendo dal primo blocco in alto a sinistra;
			\item Per ogni blocco si effettuano le seguenti attività:
				\begin{enumerate}[label=\Alph*]
					\item Si applica la dctn (DCT2 della libreria): c = DCT2(f);
					\item Vengono eliminate le frequenze ckl con k+l >= d;
					\item Si applica idctn sulla matrice c: ff = IDCT2(f)
				\end{enumerate}
			\item Trasforma da bit a immagine;
			\item Riporta l’immagine compressa sulla sezione dedicata.
	\end{enumerate}

\section{Oservazioni}
	\begin{enumerate}
		\item Le immagini richiedono di lavorare con i 4 tipi di canale: RGBA che in combinazione definiscono il colore.
	\end{enumerate}
	La traccia richiede di lavorare con immagini in toni di grigio, dalle analisi fatte(come in figura in alto a destra) si nota che il valore di \textit{un qualunque grigio} è uguale per tutti i canali.\newline
Di conseguenza si è deciso di lavorare con un solo canale (in questo caso il primo, ovvero Red).
	\begin{enumerate}
		\item Quindi l’immagine viene divisa in blocchi f di pixels di dimensione F x F, partendo dal primo blocco in alto a sinistra;
		\item Per ogni blocco si effettuano le seguenti attività:
		\begin{enumerate}[label=\Alph*]
			\item Si applica la dctn (DCT2 della libreria): c = DCT2(f);
			\item Vengono eliminate le frequenze ckl con k+l >= d;
			\item Si applica idctn sulla matrice c: ff = IDCT2(f)
		\end{enumerate}
	\end{enumerate}


\end{document}